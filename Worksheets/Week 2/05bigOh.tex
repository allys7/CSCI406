\documentclass[12pt]{article}
\usepackage{latexsym}
\usepackage{epsfig}


\setlength{\topmargin}{0in}
\setlength{\leftmargin}{0in}
\setlength{\textwidth}{6in}
\setlength{\textheight}{9.5in}
\setlength{\parindent}{0.2in}
\setlength{\parskip}{.08in}
\voffset = -.45in
\hoffset = -.5in
\def\filledbox{\vrule height 1.8ex width .8ex depth -.1ex } % square bullet
\newcommand{\qed}{\large ~$\Box$ \normalsize}
%
%\newtheorem{thm}{Theorem}
%\newenvironment{theorem}{\begin{thm}\ \rm}{\end{thm}}
%
%\newtheorem{lem}{Lemma}
%\newenvironment{lemma}{\begin{lem}\ \rm}{\end{lem}}
%
\newtheorem{theorem}{Theorem}
\newtheorem{lemma}{Lemma}
\newtheorem{corollary}{Corollary}
\newenvironment{proof}{{\noindent \bf Proof\ \ }}{\qed}
\newenvironment{proofsketch}{{\noindent {\bf Proof}\ (sketch)\ \ }}{\qed}
%
\def\shh{\skew3\hat{\hat s}}
\def\dhh{\skew6\hat{\hat d}}
\begin{document}
\newcommand{\I}{\mbox{{\em Int}}}
\newcommand{\lt}{\mbox{{\em left}}}
\newcommand{\rt}{\mbox{{\em right}}}
\newcommand{\ld}{\Delta^l}
\newcommand{\rd}{\Delta^r}
\newcommand{\lsp}[1]{\large\renewcommand{\baselinestretch}{#1}\normalsize}
\newcommand{\hsp}{\hspace{.2in}}

\def\Endwhile{\mbox{\bf endwhile\ }}
\def\Or{\mbox{\bf or\ }}
\def\Do{\mbox{\bf do\ }}
\def\Downto{\mbox{\bf downto\ }}
\def\Int{\mbox{\bf int\ }}
\def\To{\mbox{\bf to\ }}
\def\Repeat{\mbox{\bf repeat\ }}
\def\Until{\mbox{\bf until\ }}
\def\Return{\mbox{\bf return\ }}
\def\Not{\mbox{\bf not\ }}
\def\And{\mbox{\bf and\ }}
\def\For{\mbox{\bf for\ }}
\def\Foreach{\mbox{\bf foreach\ }}
\def\Else{\mbox{\bf else\ }}
\def\Elseif{\mbox{\bf elseif\ }}
\def\End{\mbox{\bf end\ }}
\def\If{\mbox{\bf if\ }}
\def\Mod{\mbox{\bf \ mod\ }}
\def\Then{\mbox{\bf then\ }}
\def\While{\mbox{\bf while\ }}
\def\Output{\mbox{\bf output\ }}


\lsp{1}
\pagestyle{plain}
\begin{center}
{\bf
Worst/Best/Average vs $O$/$\Omega$/$\Theta$
}
\end{center}

Students often confuse the concepts of worst case, best case and 
average case analysis with the three kinds of bounds ($O$,$\Omega$,$\Theta$).
The purpose of this exercise is to understand the interplay between these
two concepts. 

Suppose an algorithm takes 50, 60, 70, 80, and 100 units of 
time on inputs $A$, $B$, $C$, $D$ and $E$, respectively (and suppose these 
are the only inputs possible for the problem) 
\begin{enumerate}
\item What is the best case time of the algorithm and on which input? \\
50 units of time on input $A$.
\vspace*{0.1in}

\item What is the worst case time of the algorithm and on which input? \\
100 units of time on input $E$.
\vspace*{0.1in}

\item What is the average case time of the algorithm and on which input
(trick question)? \\
The average case time is 72 units of time over all of the inputs.
\vspace*{0.1in}

\item For the best case time (from item 1 above), provide an integer that 
\begin{enumerate}
\item serves as an upper bound ($O$ analogy), \\
\hspace{0.1in}51
\vspace*{0.1in}
\item serves as a lower bound ($\Omega$ analogy) and \\
\hspace{0.1in}49
\vspace*{0.1in}
\item {\em simultaneously} serves as an upper and lower bound ($\Theta$ analogy).  \\
\hspace{0.1in}50
\vspace*{0.1in}
\end{enumerate}

\item Why is this example an ``analogy" and not the real thing? \\
A real algorithm will almost certainly have more than five inputs
\vspace*{0.1in}

\item Repeat for the other two cases; i.e., worst (from item 2) and average
(from item 3). \\
Worst: $O(101)$, $\Omega (99)$, $\Theta(100)$.\\
Average: $O(73)$, $\Omega (71)$, $\Theta(72)$.
\end{enumerate}
\end{document} 
