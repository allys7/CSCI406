\documentclass[12pt]{article}
\usepackage{latexsym}
\usepackage{epsfig}


\setlength{\topmargin}{0in}
\setlength{\leftmargin}{0in}
\setlength{\textwidth}{6in}
\setlength{\textheight}{9.5in}
\setlength{\parindent}{0.2in}
\setlength{\parskip}{.08in}
\voffset = -.45in
\hoffset = -.5in
\def\filledbox{\vrule height 1.8ex width .8ex depth -.1ex } % square bullet
\newcommand{\qed}{\large ~$\Box$ \normalsize}
%
%\newtheorem{thm}{Theorem}
%\newenvironment{theorem}{\begin{thm}\ \rm}{\end{thm}}
%
%\newtheorem{lem}{Lemma}
%\newenvironment{lemma}{\begin{lem}\ \rm}{\end{lem}}
%
\newtheorem{theorem}{Theorem}
\newtheorem{lemma}{Lemma}
\newtheorem{corollary}{Corollary}
\newenvironment{proof}{{\noindent \bf Proof\ \ }}{\qed}
\newenvironment{proofsketch}{{\noindent {\bf Proof}\ (sketch)\ \ }}{\qed}
%
\def\shh{\skew3\hat{\hat s}}
\def\dhh{\skew6\hat{\hat d}}
\begin{document}
\newcommand{\I}{\mbox{{\em Int}}}
\newcommand{\lt}{\mbox{{\em left}}}
\newcommand{\rt}{\mbox{{\em right}}}
\newcommand{\ld}{\Delta^l}
\newcommand{\rd}{\Delta^r}
\newcommand{\lsp}[1]{\large\renewcommand{\baselinestretch}{#1}\normalsize}
\newcommand{\hsp}{\hspace{.2in}}

\def\Endwhile{\mbox{\bf endwhile\ }}
\def\Or{\mbox{\bf or\ }}
\def\Do{\mbox{\bf do\ }}
\def\Downto{\mbox{\bf downto\ }}
\def\Int{\mbox{\bf int\ }}
\def\To{\mbox{\bf to\ }}
\def\Repeat{\mbox{\bf repeat\ }}
\def\Until{\mbox{\bf until\ }}
\def\Return{\mbox{\bf return\ }}
\def\Not{\mbox{\bf not\ }}
\def\And{\mbox{\bf and\ }}
\def\For{\mbox{\bf for\ }}
\def\Foreach{\mbox{\bf foreach\ }}
\def\Else{\mbox{\bf else\ }}
\def\Elseif{\mbox{\bf elseif\ }}
\def\End{\mbox{\bf end\ }}
\def\If{\mbox{\bf if\ }}
\def\Mod{\mbox{\bf \ mod\ }}
\def\Then{\mbox{\bf then\ }}
\def\While{\mbox{\bf while\ }}
\def\Output{\mbox{\bf output\ }}


\lsp{1}
\pagestyle{plain}
\begin{center}
{\bf
Priority Queue Worksheet
}
\end{center}

\begin{enumerate}
\item
Suppose that the sequence of operations 
\begin{quote}
P R I O * R * * I * T * Y * * * Q U E * * * U * E
\end{quote}
(where a letter means {\em ``insert" that letter} and an asterisk means 
{\em ``remove the maximum"}) is applied to an initially empty
priority queue. Give the sequence of values returned by the
{\em remove the maximum} operations. 

\vspace*{2in}

\item Repeat for the scenario where the asterisk means ``{\em remove 
the minimum}". 

\end{enumerate}
\end{document} 
