\documentclass[12pt]{article}
\usepackage{latexsym}
\usepackage{epsfig}


\setlength{\topmargin}{0in}
\setlength{\leftmargin}{0in}
\setlength{\textwidth}{6in}
\setlength{\textheight}{9.5in}
\setlength{\parindent}{0.2in}
\setlength{\parskip}{.08in}
\voffset = -.45in
\hoffset = -.5in
\def\filledbox{\vrule height 1.8ex width .8ex depth -.1ex } % square bullet
\newcommand{\qed}{\large ~$\Box$ \normalsize}
\begin{document}
\newcommand{\I}{\mbox{{\em Int}}}
\newcommand{\lt}{\mbox{{\em left}}}
\newcommand{\rt}{\mbox{{\em right}}}
\newcommand{\ld}{\Delta^l}
\newcommand{\rd}{\Delta^r}
\newcommand{\lsp}[1]{\large\renewcommand{\baselinestretch}{#1}\normalsize}
\newcommand{\hsp}{\hspace{.2in}}

\lsp{1}
\pagestyle{plain}
\begin{center}
{\bf
Min-heaps
}
\end{center}

\begin{enumerate}
\item Draw the tree shape for a heap with nine elements. Don't put any
numbers inside the vertices.

\vspace*{2in} 

\item  
The smallest element in a heap must appear in position 1 and the second
smallest element must be in position 2 or 3. In your drawing above, 
mark vertices in where the third smallest element can appear with a 'C'.

\end{enumerate} 

\end{document} 
