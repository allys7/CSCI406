\documentclass[12pt]{article}
\usepackage{latexsym}
\usepackage{epsfig}


\setlength{\topmargin}{0in}
\setlength{\leftmargin}{0in}
\setlength{\textwidth}{6in}
\setlength{\textheight}{9.5in}
\setlength{\parindent}{0.2in}
\setlength{\parskip}{.08in}
\voffset = -.45in
\hoffset = -.5in
\def\filledbox{\vrule height 1.8ex width .8ex depth -.1ex } % square bullet
\newcommand{\qed}{\large ~$\Box$ \normalsize}
%
\begin{document}
\newcommand{\lsp}[1]{\large\renewcommand{\baselinestretch}{#1}\normalsize}

\lsp{1}
\pagestyle{plain}
\begin{center}
    {\bf
        Counterexample Worksheet
    }
\end{center}

The discovery of an algorithm often begins with a sudden insight into the
problem. Sometimes (unfortunately), an idea that seems very intuitive at
first glance, turns out not to be correct on further thought. Figuring
out that something is not correct by finding counterexamples is a useful
skill. Among other things, it deepens your understanding of the problem.

Find counterexamples for the following propositions:
\begin{enumerate}
    \item
          {\bf Proposition}: $a + b > \min(a,b)$

          $a = -1$, $b = -2$

          $a + b = -3$

          $min(a, b) = -2$

          $-3 < -2$
          \vspace*{0.5in}

    \item
          {\bf Proposition}: the shortest route in a road network between two points
          is one with the fewest turns.

          It is possible for a shorter road to involve more turns. For example, if a road is made up of
          three segments each a quarter-mile long, and there is another route that is two half-mile sections,
          then the shortest route would have two turns, but a longer route would exist with fewer turns.

          \vspace*{0.5in}

    \item
          {\bf Proposition}: being up by a queen in a game of chess guarantees a
          win!

          It is possible to be up by a queen and still lose in chess. For example, your opponent could be
          missing their queen, but instead use a rook and a bishop to corner your king and put you in checkmate.

          \vspace*{0.5in}

\end{enumerate}

\end{document}
