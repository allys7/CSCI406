\documentclass[12pt]{article}
\usepackage{latexsym}
\usepackage{epsfig}


\setlength{\topmargin}{0in}
\setlength{\leftmargin}{0in}
\setlength{\textwidth}{6in}
\setlength{\textheight}{9.5in}
\setlength{\parindent}{0.2in}
\setlength{\parskip}{.08in}
\voffset = -.45in
\hoffset = -.5in
\def\filledbox{\vrule height 1.8ex width .8ex depth -.1ex } % square bullet
\newcommand{\qed}{\large ~$\Box$ \normalsize}
%
%\newtheorem{thm}{Theorem}
%\newenvironment{theorem}{\begin{thm}\ \rm}{\end{thm}}
%
%\newtheorem{lem}{Lemma}
%\newenvironment{lemma}{\begin{lem}\ \rm}{\end{lem}}
%
\newtheorem{theorem}{Theorem}
\newtheorem{lemma}{Lemma}
\newtheorem{corollary}{Corollary}
\newenvironment{proof}{{\noindent \bf Proof\ \ }}{\qed}
\newenvironment{proofsketch}{{\noindent {\bf Proof}\ (sketch)\ \ }}{\qed}
%
\def\shh{\skew3\hat{\hat s}}
\def\dhh{\skew6\hat{\hat d}}
\begin{document}
\newcommand{\I}{\mbox{{\em Int}}}
\newcommand{\lt}{\mbox{{\em left}}}
\newcommand{\rt}{\mbox{{\em right}}}
\newcommand{\ld}{\Delta^l}
\newcommand{\rd}{\Delta^r}
\newcommand{\lsp}[1]{\large\renewcommand{\baselinestretch}{#1}\normalsize}
\newcommand{\hsp}{\hspace{.2in}}

\def\Endwhile{\mbox{\bf endwhile\ }}
\def\Or{\mbox{\bf or\ }}
\def\Do{\mbox{\bf do\ }}
\def\Downto{\mbox{\bf downto\ }}
\def\Int{\mbox{\bf int\ }}
\def\To{\mbox{\bf to\ }}
\def\Repeat{\mbox{\bf repeat\ }}
\def\Until{\mbox{\bf until\ }}
\def\Return{\mbox{\bf return\ }}
\def\Not{\mbox{\bf not\ }}
\def\And{\mbox{\bf and\ }}
\def\For{\mbox{\bf for\ }}
\def\Foreach{\mbox{\bf foreach\ }}
\def\Else{\mbox{\bf else\ }}
\def\Elseif{\mbox{\bf elseif\ }}
\def\End{\mbox{\bf end\ }}
\def\If{\mbox{\bf if\ }}
\def\Mod{\mbox{\bf \ mod\ }}
\def\Then{\mbox{\bf then\ }}
\def\While{\mbox{\bf while\ }}
\def\Output{\mbox{\bf output\ }}


\lsp{1}
\pagestyle{plain}
\begin{center}
    {\bf
        Insertion-Sort/Execution-Counter Worksheet
    }
\end{center}

Assume array $A$ is indexed from $1$ to $n$.

\noindent
INSERTION\_SORT($A$, $n$)\\
1. \For $j \leftarrow 2$ \To $n$ \Do\\
2. \indent \indent {\em key} $\leftarrow A[j]$\\
3. \indent \indent $i \leftarrow j-1$;\\
4. \indent \indent \While $i > 0$ \And $A[i] >$ {\em key} \Do\\
5. \indent \indent \indent \indent $A[i+1] \leftarrow A[i]$\\
6. \indent \indent \indent \indent $i \leftarrow i-1$\\
7. \indent \indent $A[i+1] \leftarrow$ {\em key}

\vspace*{0.5in}
\noindent
Instance 1 : [4, 3, 2, 1]\\
Instance 2 : [1, 4, 2, 3]\\
Instance 3 : [5, 4, 3, 2, 1]\\
Instance 4 : [1, 2, 3, 4]

\vspace*{0.5in}
\begin{tabular}{|c|r|r|r|r|} \hline
            & \multicolumn{4}{c|}{\# Times Executed}                                        \\ \hline
    Line No & Instance 1                             & Instance 2 & Instance 3 & Instance 4 \\ \hline
    L1      & 3                                      & 3          & 4          & 3          \\ \hline
    L2      & 3                                      & 3          & 4          & 3          \\ \hline
    L3      & 3                                      & 3          & 4          & 3          \\ \hline
    L4      & 8                                      & 7          & 16         & 3          \\ \hline
    L5      & 6                                      & 2          & 12         & 0          \\ \hline
    L6      & 6                                      & 2          & 12         & 0          \\ \hline
    L7      & 3                                      & 3          & 4          & 3          \\ \hline
    Total   & 32                                     & 23         & 56         & 15         \\ \hline
\end{tabular}

\noindent
List any observations.

In instance 1, the list is in reverse order and has $2n^2$ operations, while the instance that I created was the same size and already sorted, and had $n^2 -1$ steps (according to my calculations).

More generally speaking, the closer to being sorted an input was, the fewer instructions it took, as long as the size was held constant. However, even with a small increase in the length of the input the number of times each line was executed jumped dramatically.

\end{document}