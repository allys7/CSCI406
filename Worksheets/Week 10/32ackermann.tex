\documentclass[12pt]{article}
\usepackage{latexsym}
\usepackage{epsfig}
\usepackage{amsmath}
\usepackage{amssymb}


\setlength{\topmargin}{0in}
\setlength{\leftmargin}{0in}
\setlength{\textwidth}{6in}
\setlength{\textheight}{9.5in}
\setlength{\parindent}{0.2in}
\setlength{\parskip}{.08in}
\voffset = -.45in
\hoffset = -.5in
\def\filledbox{\vrule height 1.8ex width .8ex depth -.1ex } % square bullet
\newcommand{\qed}{\large ~$\Box$ \normalsize}
%
%\newtheorem{thm}{Theorem}
%\newenvironment{theorem}{\begin{thm}\ \rm}{\end{thm}}
%
%\newtheorem{lem}{Lemma}
%\newenvironment{lemma}{\begin{lem}\ \rm}{\end{lem}}
%
\newtheorem{theorem}{Theorem}
\newtheorem{lemma}{Lemma}
\newtheorem{corollary}{Corollary}
\newenvironment{proof}{{\noindent \bf Proof\ \ }}{\qed}
\newenvironment{proofsketch}{{\noindent {\bf Proof}\ (sketch)\ \ }}{\qed}
%
\def\shh{\skew3\hat{\hat s}}
\def\dhh{\skew6\hat{\hat d}}
\begin{document}
\newcommand{\I}{\mbox{{\em Int}}}
\newcommand{\lt}{\mbox{{\em left}}}
\newcommand{\rt}{\mbox{{\em right}}}
\newcommand{\ld}{\Delta^l}
\newcommand{\rd}{\Delta^r}
\newcommand{\lsp}[1]{\large\renewcommand{\baselinestretch}{#1}\normalsize}
\newcommand{\hsp}{\hspace{.2in}}

\def\Endwhile{\mbox{\bf endwhile\ }}
\def\Or{\mbox{\bf or\ }}
\def\Do{\mbox{\bf do\ }}
\def\Downto{\mbox{\bf downto\ }}
\def\Int{\mbox{\bf int\ }}
\def\To{\mbox{\bf to\ }}
\def\Repeat{\mbox{\bf repeat\ }}
\def\Until{\mbox{\bf until\ }}
\def\Return{\mbox{\bf return\ }}
\def\Not{\mbox{\bf not\ }}
\def\And{\mbox{\bf and\ }}
\def\For{\mbox{\bf for\ }}
\def\Foreach{\mbox{\bf foreach\ }}
\def\Else{\mbox{\bf else\ }}
\def\Elseif{\mbox{\bf elseif\ }}
\def\End{\mbox{\bf end\ }}
\def\If{\mbox{\bf if\ }}
\def\Mod{\mbox{\bf \ mod\ }}
\def\Then{\mbox{\bf then\ }}
\def\While{\mbox{\bf while\ }}
\def\Output{\mbox{\bf output\ }}


\lsp{1}
\pagestyle{plain}
\begin{center}
{\bf
Ackermann's Function
}
\end{center}

The Ackermann's function is defined by the following recurrence
relation:\\
$A(1,j) = 2^j$ for $j \ge 1$\\
$A(i,1) = A(i-1,2)$ for $i \ge 2$\\
$A(i,j) = A(i-1,A(i,j-1))$ for $i,j \ge 2$

\vspace*{0.5in}
Use the recurrence relation to fill up as many values as you can 
in the table below. Start with Row 1 and work your way up to 
larger values of $i$ and $j$.
\vspace*{0.5in}

%\begin{table}{h}
\begin{tabular}{|c|c|c|c|c|c}\hline
\multicolumn{6}{|c}{Ackermann Table}\\\hline
$i/j$ & 1 & 2 & 3 & 4 & $\cdots$ \\ \hline
1   &   $2^1$ &   $2^2$ &   $2^3$ &  $2^4$ &     \\
    &     &     &     &     &     \\  \hline
2   &   $2^2$ &  $2^4$ &   $2^{16}$  &    $2^{65536}$ &     \\
    &     &     &     &     &     \\ \hline
3   &  $2^4$ &  &   &   &     \\
    &    &  &   &   &     \\ \hline
... &   \hspace*{0.5in}  & \hspace*{0.5in}  &  \hspace*{0.5in}  &  \hspace*{0.5in}  &     \\
\end{tabular}
%\end{table}

\vspace*{0.5in}
What pattern emerges in Row 2?

Each element becomes $2^X$ where $X$ is the cell to its left. So from $2^2$ we go to $2^{2^2}$ to $2^{2^{2^2}}$, etc.

\end{document}
