\documentclass[12pt]{article}
\usepackage{latexsym}
\usepackage{epsfig}


\setlength{\topmargin}{0in}
\setlength{\leftmargin}{0in}
\setlength{\textwidth}{6in}
\setlength{\textheight}{9.5in}
\setlength{\parindent}{0.2in}
\setlength{\parskip}{.08in}
\voffset = -.45in
\hoffset = -.5in
\def\filledbox{\vrule height 1.8ex width .8ex depth -.1ex } % square bullet
\newcommand{\qed}{\large ~$\Box$ \normalsize}
%
%\newtheorem{thm}{Theorem}
%\newenvironment{theorem}{\begin{thm}\ \rm}{\end{thm}}
%
%\newtheorem{lem}{Lemma}
%\newenvironment{lemma}{\begin{lem}\ \rm}{\end{lem}}
%
\newtheorem{theorem}{Theorem}
\newtheorem{lemma}{Lemma}
\newtheorem{corollary}{Corollary}
\newenvironment{proof}{{\noindent \bf Proof\ \ }}{\qed}
\newenvironment{proofsketch}{{\noindent {\bf Proof}\ (sketch)\ \ }}{\qed}
%
\def\shh{\skew3\hat{\hat s}}
\def\dhh{\skew6\hat{\hat d}}
\begin{document}
\newcommand{\I}{\mbox{{\em Int}}}
\newcommand{\lt}{\mbox{{\em left}}}
\newcommand{\rt}{\mbox{{\em right}}}
\newcommand{\ld}{\Delta^l}
\newcommand{\rd}{\Delta^r}
\newcommand{\lsp}[1]{\large\renewcommand{\baselinestretch}{#1}\normalsize}
\newcommand{\hsp}{\hspace{.2in}}

\def\Endwhile{\mbox{\bf endwhile\ }}
\def\Or{\mbox{\bf or\ }}
\def\Do{\mbox{\bf do\ }}
\def\Downto{\mbox{\bf downto\ }}
\def\Int{\mbox{\bf int\ }}
\def\To{\mbox{\bf to\ }}
\def\Repeat{\mbox{\bf repeat\ }}
\def\Until{\mbox{\bf until\ }}
\def\Return{\mbox{\bf return\ }}
\def\Not{\mbox{\bf not\ }}
\def\And{\mbox{\bf and\ }}
\def\For{\mbox{\bf for\ }}
\def\Foreach{\mbox{\bf foreach\ }}
\def\Else{\mbox{\bf else\ }}
\def\Elseif{\mbox{\bf elseif\ }}
\def\End{\mbox{\bf end\ }}
\def\If{\mbox{\bf if\ }}
\def\Mod{\mbox{\bf \ mod\ }}
\def\Then{\mbox{\bf then\ }}
\def\While{\mbox{\bf while\ }}
\def\Output{\mbox{\bf output\ }}


\lsp{1}
\pagestyle{plain}
\hfill Ally Smith
\begin{center}
{\bf
% Worksheet title here %
Homework 1
}
\end{center}

\begin{enumerate}
\item Prove the correctness of the recursive algorithm

\vspace{0.1in}
\item Prove by induction that for $n \geq 1$, $\sum_{i=1} ^{n}{\frac{1}{i(i+1)}} = \frac{n}{n+1}$

Base case $(n=1)$: $\frac{1}{1(1+1)} = \frac{1}{2}$\\
Assume that $\sum_{i=1} ^{n}{\frac{1}{i(i+1)}} = \frac{1}{2} + \frac{1}{6} + . . . + \frac{1}{n(n+1)} = \frac{n}{n+1}$\\
Therefore, we can assume $\sum_{i=1} ^{n+1}{\frac{1}{i(i+1)}} = \frac{1}{2} + \frac{1}{6} + . . . + \frac{1}{n(n+1)} + \frac{1}{(n+1)(n+2)}= \frac{n+1}{n+2}$\\
Substitute in from our previous assumption , and the rest is algebra.\\

$\frac{n}{n+1} + \frac{1}{(n+1)(n+2)}= \frac{n+1}{n+2}$\\
$\frac{n(n+2)}{(n+1)(n+2)} + \frac{1}{(n+1)(n+2)}= \frac{n+1}{n+2}$\\
$\frac{n^2+2n+1}{(n+1)(n+2)}= \frac{n+1}{n+2}$\\
$\frac{(n+1)^2}{(n+1)(n+2)}= \frac{n+1}{n+2}$
$\frac{n+1}{n+2} = \frac{n+1}{n+2}$\\

Therefore, we have proven that for any case $n+1$ the summation holds true, proving our original statement.
\vspace{0.1in}
\item A sorting algorithm takes 1 second to sort 1,000 items on your local machine. \ 
How long will it take to sort 10,000 items. . .
    \begin{enumerate}
        \item if you believe that the algorithm takes time proportional to $n^2$, and\vspace{0.05in}\\
            $C(1,000)^2=1$ second \vspace{0.05in}\\
            $C = \frac{1}{1,000^2}$ \vspace{0.05in}\\
            $C(10,000)^2 = \frac{1}{1,000,000} \times 100,000,000 =$ {\bf 100 seconds} \\
        \item if you believe that the algorithm takes time roughly proportional to $n \log n$.
    \end{enumerate}
\item There are 25 horses. At most, 5 horses can race together at a time. You must
determine the fastest, second fastest, and third fastest horses. Find the minimum
number of races in which this can be done.

Begin by dividing the horses into five groups, with each horse assigned to a group lettered A-E.
Race these groups and throw out the bottom two horses from each race, as they will
never be fast enough to podium. \\
Our next race will be between the winners of our first five races. This gets us our
fastest horse, which we will say came from group A, and because it was the fastest
in that group, it will be called $a_1$. This race also allows us to eliminate the
groups from which the last two horses from the 6th race came, call them D and E.
This is because we have three horses, $a_1, b_1,$ and $ c_1$that are faster than
the fastest horses from those groups, $d_1$ and $e_1$. \\
In our final race, we will choose $a_2, a_3, b_1, b_2,$ and $c_1$ as our racers,
giving us the results of our fastest 3 horses, and we accomplished this in {\bf 7 races}.


\end{enumerate}

\end{document} 
