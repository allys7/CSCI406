\documentclass[12pt]{article}
\usepackage{latexsym}
\usepackage{epsfig}


\setlength{\topmargin}{0in}
\setlength{\leftmargin}{0in}
\setlength{\textwidth}{6in}
\setlength{\textheight}{9.5in}
\setlength{\parindent}{0.2in}
\setlength{\parskip}{.08in}
\voffset = -.45in
\hoffset = -.5in
\def\filledbox{\vrule height 1.8ex width .8ex depth -.1ex } % square bullet
\newcommand{\qed}{\large ~$\Box$ \normalsize}
%
%\newtheorem{thm}{Theorem}
%\newenvironment{theorem}{\begin{thm}\ \rm}{\end{thm}}
%
%\newtheorem{lem}{Lemma}
%\newenvironment{lemma}{\begin{lem}\ \rm}{\end{lem}}
%
\newtheorem{theorem}{Theorem}
\newtheorem{lemma}{Lemma}
\newtheorem{corollary}{Corollary}
\newenvironment{proof}{{\noindent \bf Proof\ \ }}{\qed}
\newenvironment{proofsketch}{{\noindent {\bf Proof}\ (sketch)\ \ }}{\qed}
%
\def\shh{\skew3\hat{\hat s}}
\def\dhh{\skew6\hat{\hat d}}
\begin{document}
\newcommand{\I}{\mbox{{\em Int}}}
\newcommand{\lt}{\mbox{{\em left}}}
\newcommand{\rt}{\mbox{{\em right}}}
\newcommand{\ld}{\Delta^l}
\newcommand{\rd}{\Delta^r}
\newcommand{\lsp}[1]{\large\renewcommand{\baselinestretch}{#1}\normalsize}
\newcommand{\hsp}{\hspace{.2in}}

\def\Endwhile{\mbox{\bf endwhile\ }}
\def\Or{\mbox{\bf or\ }}
\def\Do{\mbox{\bf do\ }}
\def\Downto{\mbox{\bf downto\ }}
\def\Int{\mbox{\bf int\ }}
\def\To{\mbox{\bf to\ }}
\def\Repeat{\mbox{\bf repeat\ }}
\def\Until{\mbox{\bf until\ }}
\def\Return{\mbox{\bf return\ }}
\def\Not{\mbox{\bf not\ }}
\def\And{\mbox{\bf and\ }}
\def\For{\mbox{\bf for\ }}
\def\Foreach{\mbox{\bf foreach\ }}
\def\Else{\mbox{\bf else\ }}
\def\Elseif{\mbox{\bf elseif\ }}
\def\End{\mbox{\bf end\ }}
\def\If{\mbox{\bf if\ }}
\def\Mod{\mbox{\bf \ mod\ }}
\def\Then{\mbox{\bf then\ }}
\def\While{\mbox{\bf while\ }}
\def\Output{\mbox{\bf output\ }}


\lsp{1}
\pagestyle{plain}
\hfill Ally Smith
\begin{center}
{\bf
% Worksheet title here %
Homework 2
}
\end{center}

\begin{enumerate}
\item True or false?
    \begin{enumerate}
        \item Is $2^{n+1} = O(2^n)$? {\bf True}
        \item Is $2^{2n} = O(2^n)$? {\bf False}
    \end{enumerate}

\item Answer the following
    \begin{enumerate}
        \item If I prove that an algorithm takes $O(n^2)$ worst-case time,
        is it possible that it takes $O(n)$ on some inputs?

        Yes. For example, a sorting algorithm given a list in reverse order may take $O(n^2)$,
        while when provided a sorted list it is only $O(n)$.

        \item If I prove that an algorithm takes $O(n^2)$ worst-case time,
        is it possible that it takes $O(n)$ on all inputs?

        Yes. Because Big Oh notation offers an upper limit, it is possible that
        the proven $O(n^2)$ is true while also being true that it is also $O(n)$.

        \item If I prove that an algorithm takes $\Theta (n^2)$ worst-case time,
        is it possible that it takes $O(n)$ on some inputs?

        Yes. Because it is $\Theta (n^2)$ of the worst case, it is reasonable to
        assume that there could be inputs that are $O(n)$.

        \item If I prove that an algorithm takes $\Theta (n^2)$ worst-case time,
        is it possible that it takes $O(n)$ on all inputs?

        No. Because the algorithm is $\Theta (n^2)$ there must be an input that
        takes $O(n^2)$.
    \end{enumerate}

    \item Assume $n$ is even. Let $T(n)$ denote the number of times ‘foobar’ is
    printed as a function of $n$.

        \begin{enumerate}
            \item Express $T(n)$ as three nested summations.
            
            $T(n) = \sum_{i=1}^{n} \sum_{j=i}^{2i} 1$

            \item Simplify the summation. Show your work.
                \begin{enumerate}
                    \item $T(n) = \sum_{i=1}^{n} \sum_{j=i}^{2i} 1$
                    \item $T(n) = \sum_{i=1}^{n} \sum_{j=i}^{2i} 1$
                    \item $T(n) = \sum_{i=1}^{n} \sum_{j=i-i}^{2i-i} 1$
                    \item $T(n) = \sum_{i=1}^{n} \sum_{j=0}^{i} 1$
                    \item $T(n) = \sum_{i=1}^{n}( i + 1)$
                    \item $T(n) = \sum_{i=1}^{n} i + \sum_{i=1}^{n} 1$
                    \item $T(n) = \frac{n(n+1)}{2} + n$
                    \item $T(n) = \frac{3n(n+1)}{2}$
                \end{enumerate}
        \end{enumerate}

    \item Prove the following identities on logarithms:
        \begin{enumerate}
            \item $log_a(xy) = log_a(x) + log_a(y)$
                \begin{enumerate}
                    \item $m = log_a (x) \rightarrow x = a^m$
                    \item $n = log_a (y) \rightarrow y = a^n$
                    \item $xy = a^m a^n = a^{m+n}$
                    \item $log_a (xy) = log_a(a^{m+n})$
                    \item $log_a (xy) = (m + n)log_a(a)$
                    \item $log_a (xy) = log_a(x) + log_a(y)$
                \end{enumerate}
            \item $log_a(x^y) = y\ log_a(x)$
                \begin{enumerate}
                    \item Let $m = log_a (x)$
                    \item Using the definition of logarithms, we can rewrite this as $x = a^m$
                    \item Now, $x^y$ can be rewritten as $(a^m)^y$
                    \item We can rewrite this as $log_a (x^y) = ym$
                    \item Finally, substitute back in our value for $m$, $log_a (x^y) = y\ log_a (x)$
                \end{enumerate}
            \item $log_a(x) = \frac{log_b (x)}{log_b (a)}$
                \begin{enumerate}
                    \item $log_a(x) = y \rightarrow a^y = x$
                    \item $log_b(a^y) = log_b(x)$
                    \item $y\ log_b(a) = log_b(x)$
                    \item $y = \frac{log_b(x)}{log_b(a)}$
                    \item $log_a(x) = \frac{log_b(x)}{log_b(a)}$
                \end{enumerate}
            \item $x^{log_b (y)} = y^{log_b (x)}$
                \begin{enumerate}
                    \item $x = b^{log_b(x)}$
                    \item $y = b^{log_b(y)}$
                    \item $x^{log_b(y)} = b^{log_b(x)log_b(y)}$
                    \item $y^{log_b(x)} = b^{log_b(x)log_b(y)}$
                    \item $x^{log_b (y)} = y^{log_b (x)}$
                \end{enumerate}
        \end{enumerate}
    \item You are given 10 bags of gold coins. Nine bags contain coins that
    each weigh 10 grams. One bag contains all false coins that weigh one gram
    less. You must identify this bag in just one weighing. You have a digital
    balance that reports the weight of what is placed on it.

    Begin by labeling your bags 1 through 10. From the first bag, remove one
    coin. From the second bag, two, etc. Place all of these coins on the scale.
    Because all of the legitimate bags weigh multiples of 10 grams, we can use
    the ones digit of the weight to determine which bag the faulty coins came
    from. For example, if Bag 3 had the counterfeit coins, the weight would end
    in a 7 because $3 \times 9 = 2${\bf 7}. This can be repeated for every
    number from 1 to 10, allowing you to determine which bag has the
    counterfeit coins.


\end{enumerate}

\end{document} 
