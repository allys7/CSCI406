\documentclass[12pt]{article}
\usepackage{latexsym}
\usepackage{epsfig}
\usepackage{tabto}


\setlength{\topmargin}{0in}
\setlength{\leftmargin}{0in}
\setlength{\textwidth}{6in}
\setlength{\textheight}{9.5in}
\setlength{\parindent}{0.2in}
\setlength{\parskip}{.08in}
\voffset = -.45in
\hoffset = -.5in
\def\filledbox{\vrule height 1.8ex width .8ex depth -.1ex } % square bullet
\newcommand{\qed}{\large ~$\Box$ \normalsize}
%
%\newtheorem{thm}{Theorem}
%\newenvironment{theorem}{\begin{thm}\ \rm}{\end{thm}}
%
%\newtheorem{lem}{Lemma}
%\newenvironment{lemma}{\begin{lem}\ \rm}{\end{lem}}
%
\newtheorem{theorem}{Theorem}
\newtheorem{lemma}{Lemma}
\newtheorem{corollary}{Corollary}
\newenvironment{proof}{{\noindent \bf Proof\ \ }}{\qed}
\newenvironment{proofsketch}{{\noindent {\bf Proof}\ (sketch)\ \ }}{\qed}
%
\def\shh{\skew3\hat{\hat s}}
\def\dhh{\skew6\hat{\hat d}}
\begin{document}
\newcommand{\I}{\mbox{{\em Int}}}
\newcommand{\lt}{\mbox{{\em left}}}
\newcommand{\rt}{\mbox{{\em right}}}
\newcommand{\ld}{\Delta^l}
\newcommand{\rd}{\Delta^r}
\newcommand{\lsp}[1]{\large\renewcommand{\baselinestretch}{#1}\normalsize}
\newcommand{\hsp}{\hspace{.2in}}

\def\Endwhile{\mbox{\bf endwhile\ }}
\def\Or{\mbox{\bf or\ }}
\def\Do{\mbox{\bf do\ }}
\def\Downto{\mbox{\bf downto\ }}
\def\Int{\mbox{\bf int\ }}
\def\To{\mbox{\bf to\ }}
\def\Repeat{\mbox{\bf repeat\ }}
\def\Until{\mbox{\bf until\ }}
\def\Return{\mbox{\bf return\ }}
\def\Not{\mbox{\bf not\ }}
\def\And{\mbox{\bf and\ }}
\def\For{\mbox{\bf for\ }}
\def\Foreach{\mbox{\bf foreach\ }}
\def\Else{\mbox{\bf else\ }}
\def\Elseif{\mbox{\bf elseif\ }}
\def\End{\mbox{\bf end\ }}
\def\If{\mbox{\bf if\ }}
\def\Mod{\mbox{\bf \ mod\ }}
\def\Then{\mbox{\bf then\ }}
\def\While{\mbox{\bf while\ }}
\def\Output{\mbox{\bf output\ }}
\def\In{\mbox{\bf in\ }}
\def\True{\mbox{\bf True\ }}
\def\False{\mbox{\bf False\ }}


\lsp{1}
\pagestyle{plain}
\hfill Ally Smith
\begin{center}
    {\bf Homework 3}
\end{center}

\begin{enumerate}
    \item Design a dictionary data structure in which search, insertion, and deletion can all be
          processed in $O(1)$ time in the worst case. You may assume the set elements are integers drawn
          from a finite set $1, 2, .., n$, and initialization can take $O(n)$ time.

          We can implement this dictionary using a hash table, which will allow us to have constant time
          for our search, insert, and delete operations. The search is done using a simple hash table
          lookup, and the insert/delete operations rely on this table lookup also. In the delete, it is
          used to find the value to be deleted, and in the insert it is used to check if the value already
          exists in the set. With this definition, our initialization would take $O(n)$ since it would be
          $O(1)$ for each of the $n$ elements.

    \item Describe how to modify any balanced tree data structure such that search, insert, delete,
          minimum, and maximum still take $O(\log n)$ time each, but successor and predecessor now take $O(1)$
          time each. Which operations have to be modified to support this?

          A balanced binary tree can be modified by adding additional pointers within each node
          to its respective successor and predecessor. By storing this data, we can access these
          nodes in constant time. However, this would require us to modify the insert and delete
          operations so that these pointers stay updated as the tree grows. When inserting a node,
          we need to adjust the nodes that are the successor and predecessor of the inserted node.
          When we delete a node, we need to adjust the successor and predecessor to go over the
          deleted node, effectively removing it.
          \pagebreak
    \item In the {\em bin-packing problem}, we are given $n$ metal objects, each weighing between zero and one
          kilogram. Our goal is to find the smallest number of bins that will hold the $n$ objects, with each
          bin holding one kilogram at most.

          The {\em best-fit heuristic} for bin packing is as follows. Consider the objects in the order in which they are
          given. For each object, place it into the partially filled bin with the smallest amount of extra room
          after the object is inserted.. If no such bin exists, start a new bin. Design an algorithm that
          implements the best-fit heuristic (taking as input the $n$ weights $w_1, w_2,\ ...\ , w_n$ and outputting the
          number of bins used) in $O(n\log n)$ time.

          \tabto{0cm}bins = []
          \tabto{0cm}\For\ object\ \In objects:
          \tabto{1cm}minSpaceRemaining = $\infty$
          \tabto{1cm}binIndex = -1
          \tabto{1cm}\For bin\ \In existingBins:
          \tabto{2cm} currentSpaceRemaining = bin.spaceRemaining - object.mass
          \tabto{2cm} \If currentSpaceRemaining $<$ minSpaceRemaining:
          \tabto{3cm} minSpaceRemaining = currentSpaceRemaining
          \tabto{3cm} binIndex = bin.index
          \tabto{1cm} \If minSpaceRemaining == $\infty$:
          \tabto{2cm} // case for items not fitting into an existing bin
          \tabto{2cm} create new bin
          \tabto{2cm} add current item to bin
          \tabto{2cm} add bin to bins
          \tabto{1cm} \Else:
          \tabto{2cm} // general case for adding to best-fit bin
          \tabto{2cm} add item to bins[binIndex]
          \tabto{0cm}\Return bins.size

    \item Determine whether a linked list contains a loop as quickly as possible
          without using any extra storage. Also, identify the location of the loop.

          This can be implemented without using any additional storage by simulatneously
          looping through the list at two different speeds. If the values in these fast
          and slow loops intersect, then there is a loop in the list. If either of them
          reach $null$ then we know that there isn't a loop. Pseudocode below:

          \tabto{0cm}boolean hasLoop(Node head):
          \tabto{1cm}slow = fast = head
          \tabto{1cm}$\While \True$:
          \tabto{2cm}slow = slow.next
          \tabto{2cm}$\If$ fast.next $\neq null$:
          \tabto{3cm}fast = fast.next.next;
          \tabto{2cm}\Else \Return \False

          \tabto{2cm}\If slow == $null$ \Or fast == $null$:
          \tabto{3cm}\Return \False
          \tabto{2cm}\If slow == fast:
          \tabto{3cm}\Return \True

\end{enumerate}
\end{document}
