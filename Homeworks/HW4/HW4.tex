\documentclass[12pt]{article}
\usepackage{latexsym}
\usepackage{epsfig}
\usepackage[edges]{forest}
\usepackage{tabto}
\usepackage{amsmath}


\setlength{\topmargin}{0in}
\setlength{\leftmargin}{0in}
\setlength{\textwidth}{6in}
\setlength{\textheight}{9.5in}
\setlength{\parindent}{0.2in}
\setlength{\parskip}{.08in}
\voffset = -.45in
\hoffset = -.5in
\def\filledbox{\vrule height 1.8ex width .8ex depth -.1ex } % square bullet
\newcommand{\qed}{\large ~$\Box$ \normalsize}
%
%\newtheorem{thm}{Theorem}
%\newenvironment{theorem}{\begin{thm}\ \rm}{\end{thm}}
%
%\newtheorem{lem}{Lemma}
%\newenvironment{lemma}{\begin{lem}\ \rm}{\end{lem}}
%
\newtheorem{theorem}{Theorem}
\newtheorem{lemma}{Lemma}
\newtheorem{corollary}{Corollary}
\newenvironment{proof}{{\noindent \bf Proof\ \ }}{\qed}
\newenvironment{proofsketch}{{\noindent {\bf Proof}\ (sketch)\ \ }}{\qed}
%
\def\shh{\skew3\hat{\hat s}}
\def\dhh{\skew6\hat{\hat d}}
\begin{document}
\newcommand{\I}{\mbox{{\em Int}}}
\newcommand{\lt}{\mbox{{\em left}}}
\newcommand{\rt}{\mbox{{\em right}}}
\newcommand{\ld}{\Delta^l}
\newcommand{\rd}{\Delta^r}
\newcommand{\lsp}[1]{\large\renewcommand{\baselinestretch}{#1}\normalsize}
\newcommand{\hsp}{\hspace{.2in}}

\def\Endwhile{\mbox{\bf endwhile\ }}
\def\Or{\mbox{\bf or\ }}
\def\Do{\mbox{\bf do\ }}
\def\Downto{\mbox{\bf downto\ }}
\def\Int{\mbox{\bf int\ }}
\def\To{\mbox{\bf to\ }}
\def\Repeat{\mbox{\bf repeat\ }}
\def\Until{\mbox{\bf until\ }}
\def\Return{\mbox{\bf return\ }}
\def\Not{\mbox{\bf not\ }}
\def\And{\mbox{\bf and\ }}
\def\For{\mbox{\bf for\ }}
\def\Foreach{\mbox{\bf foreach\ }}
\def\Else{\mbox{\bf else\ }}
\def\Elseif{\mbox{\bf elseif\ }}
\def\End{\mbox{\bf end\ }}
\def\If{\mbox{\bf if\ }}
\def\Mod{\mbox{\bf \ mod\ }}
\def\Then{\mbox{\bf then\ }}
\def\While{\mbox{\bf while\ }}
\def\Output{\mbox{\bf output\ }}


\lsp{1}
\pagestyle{plain}
\hfill Ally Smith
\begin{center}
{\bf
% Worksheet title here %
Homework 1
}
\end{center}

\begin{enumerate}
    \item For each of the following problems, give an algorithm that finds the desired
    numbers within the given amount of time.
    \begin{enumerate}
        \item Let $S$ be an \emph{unsorted} array of $n$ integers. Give an algorithm
        that finds the pair $x, y \in S$ that \emph{maximizes} $|x-y|$. Your algorithm must
        run in $O(n)$ worst-case time.

        \tabto{1cm}let $min = 0, max = \infty$
        \tabto{1cm}$\For value \in S$
        \tabto{2cm}$\If value < min\ \Then min = value$
        \tabto{2cm}$\If value > max\ \Then max = value$
        \tabto{1cm}$\Return (max, min)$

        \item Let $S$ be a \emph{sorted} array of $n$ integers. Give an algorithm that finds the pair
        $x, y \in S$ that \emph{maximizes} $|x-y|$. Your algorithm must run in $O(1)$ worst-case time.

        \tabto{1cm}let $max = S[n-1]$
        \tabto{1cm}let $min = S[0]$
        \tabto{1cm}$\Return (max, min)$

        \item  Let $S$ be an \emph{unsorted} array of $n$ integers. Give an algorithm that finds the pair
        $x, y \in S$ that \emph{minimizes} $|x-y|$, for $x\neq y$. Your algorithm must run in $O(n log n)$
        worst-case time.

        \tabto{1cm}Sort $S$ using heapsort
        \tabto{1cm}Follow the algorithm used in part d.

        \item Let $S$ be a \emph{sorted} array of $n$ integers. Give an algorithm that finds the pair
        $x, y \in S$ that \emph{minimizes} $|x-y|$, for $x \neq y$. Your algorithm must run in $O(n)$
        worst-case time.

        \tabto{1cm}let index1 $= 0$, index2 $= 0, $minDifference $= \infty$
        \tabto{1cm}$\For i \in [0,n-1)$
        \tabto{2cm}$\If (S$[$i+1$]$-S$[$i$]$)<$minDifference:
        \tabto{3cm}$\Then$ index1$=i+1$, index2$=i$
        \tabto{1cm}$\Return $($S$[index1], $S$[index2])
    \end{enumerate}
    \pagebreak
    \item Given two sets $S_1$ and $S_2$ (each of size $n$), and a number $x$, describe an $O(n \log n)$
    algorithm for finding whether there exists a pair of elements, one from $S_1$ and one
    from $S_2$, that add up to $x$. (For partial credit, give a $\Theta(n^2)$ algorithm for this
    problem.)

    \tabto{1cm}Sort $S_1$ using heapsort\tabto{11cm}$O(n \log n)$
    \tabto{1cm}$\For n \in S_2$\tabto{11cm}$O(n)$
    \tabto{2cm}let diff $= x-n$\tabto{11cm}$O(1)$
    \tabto{2cm}Perform a binary search on $S_1$ for diff\tabto{11cm}$O(\log n)$
    \tabto{2cm}$\If$ it is found $\Then \Return ($diff, n)\tabto{11cm}$O(1)$

    This is $O(n \log n) + O(n\log n) = O(n\log n)$

    \item Devise an algorithm for finding the $k$ smallest elements of an \emph{unsorted} set of $n$
    integers in $O(n + k \log n)$.  

    Using a max-heap limited to size $k$, we can loop through each item $O(n)$ in the set and compare it to the
    maximum element in the heap $O(1)$. If the element is less than the max, then replace the maximum in the heap
    $O(1)$. At the end, the elements in the max-heap will be the k smallest elements. This is $O(n)$.

    \item Mr. B. C. Dull claims to have developed a new data structure for priority queues
    that supports the operations Insert, Maximum, and Extract-Max — all in $O(1)$ worst-case time.
    Prove that he is mistaken.

    \item Suppose that you are given a sorted sequence of distinct integers ${a_1, a_2,...,a_n}$.
    Give an $O(\log n)$ algorithm to determine whether there exists an $i$ index such as $a_i = i$.
    For example, in $\{-10, -3, 3, 5, 7\}$, $a_3 = 3$. In $\{2, 3, 4, 5, 6, 7\}$, there is no such $i$.
\end{enumerate}
\end{document} 
