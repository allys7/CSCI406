\documentclass[12pt]{article}
\usepackage{latexsym}
\usepackage{epsfig}


\setlength{\topmargin}{0in}
\setlength{\leftmargin}{0in}
\setlength{\textwidth}{6in}
\setlength{\textheight}{9.5in}
\setlength{\parindent}{0.2in}
\setlength{\parskip}{.08in}
\voffset = -.45in
\hoffset = -.5in
\def\filledbox{\vrule height 1.8ex width .8ex depth -.1ex } % square bullet
\newcommand{\qed}{\large ~$\Box$ \normalsize}
%
%\newtheorem{thm}{Theorem}
%\newenvironment{theorem}{\begin{thm}\ \rm}{\end{thm}}
%
%\newtheorem{lem}{Lemma}
%\newenvironment{lemma}{\begin{lem}\ \rm}{\end{lem}}
%
\newtheorem{theorem}{Theorem}
\newtheorem{lemma}{Lemma}
\newtheorem{corollary}{Corollary}
\newenvironment{proof}{{\noindent \bf Proof\ \ }}{\qed}
\newenvironment{proofsketch}{{\noindent {\bf Proof}\ (sketch)\ \ }}{\qed}
%
\def\shh{\skew3\hat{\hat s}}
\def\dhh{\skew6\hat{\hat d}}
\begin{document}
\newcommand{\I}{\mbox{{\em Int}}}
\newcommand{\lt}{\mbox{{\em left}}}
\newcommand{\rt}{\mbox{{\em right}}}
\newcommand{\ld}{\Delta^l}
\newcommand{\rd}{\Delta^r}
\newcommand{\lsp}[1]{\large\renewcommand{\baselinestretch}{#1}\normalsize}
\newcommand{\hsp}{\hspace{.2in}}

\def\Endwhile{\mbox{\bf endwhile\ }}
\def\Or{\mbox{\bf or\ }}
\def\Do{\mbox{\bf do\ }}
\def\Downto{\mbox{\bf downto\ }}
\def\Int{\mbox{\bf int\ }}
\def\To{\mbox{\bf to\ }}
\def\Repeat{\mbox{\bf repeat\ }}
\def\Until{\mbox{\bf until\ }}
\def\Return{\mbox{\bf return\ }}
\def\Not{\mbox{\bf not\ }}
\def\And{\mbox{\bf and\ }}
\def\For{\mbox{\bf for\ }}
\def\Foreach{\mbox{\bf foreach\ }}
\def\Else{\mbox{\bf else\ }}
\def\Elseif{\mbox{\bf elseif\ }}
\def\End{\mbox{\bf end\ }}
\def\If{\mbox{\bf if\ }}
\def\Mod{\mbox{\bf \ mod\ }}
\def\Then{\mbox{\bf then\ }}
\def\While{\mbox{\bf while\ }}
\def\Output{\mbox{\bf output\ }}


\lsp{1}
\pagestyle{plain}
\hfill Ally Smith
\begin{center}
{\bf
% Worksheet title here %
Knapsack Report
}
\end{center}

\begin{enumerate}
\item For the exhaustive approach, I implemented a solution outlined by Simon Hessner in order to obtain the powerset of the items.\
We begin by iterating $i$ over $[0, 2^n)$ to account for the fact that for any set of $n$ elements, the power set contains $2^n$ elements.\
The binary representation of $i$ can be interpreted bit by bit to determine each combination of elements in the powerset.\
For example, if $i = 1001_2$, then this would mean that the 0th and 3rd elements would be in this subset of the powerset.\
In order to evaluate this, we iterate $k$ from $[0, n)$ and use $k$ as the shift amount for the bitwise shift in the expression $i\ \&\ 1<<k$.\
If this expression evaluates to true, that means that there is a one bit in the $k$th index, implying that index belongs in the subset.\
From here, we can add items to our subset, and when our $k$ loop finishes, we simply add this subset to the powerset.\
Because this algorithm relies primarily on bitwise operations within the loops, we are able to fairly quickly obtain the powerset of any set of items.

For the greedy heuristic approach, I relied on the built-in $sorted$ function in Python. This function uses the Timsort algorithm,\
which has an average and worst case performance of $O(n\ log(n))$. 

\item 


\end{enumerate}

{\bf
\begin{center}
    Appendix
\end{center}
}
\begin{verbatim}
# Project 1 - Knapsack problem
# Ally Smith
# Sept. 9, 2021
# CSCI 406
from random import randint

# basic item class that stores the weight, value, and their ratio
class Item:
    def __init__(self, w, v):
        self.weight = w
        self.value = v
        self.ratio = w/v

    def __repr__(self) -> str:
        string = "(weight=" + str(self.weight) + ", value=" + \
            str(self.value) + ", ratio=" + str(self.ratio) + ")"
        return string


# function to get the powerset of a given set 
# source for this elegant solution:
# https://simonhessner.de/calculate-power-set-set-of-all
# -subsets-in-python-without-recursion/
def get_powerset(list):
    n = len(list)
    return [[list[k] for k in range(n) if i&1<<k] for i in range(2**n)]

# exhaustive approach, from provided pseudocode
def exhaustive(W, n, items):
    knapsack = []
    best_value = 0
    
    powerset = get_powerset(items)
    for subset in powerset:
        subset_value = 0
        subset_weight = 0
        for item in subset:
            subset_value += item.value
            subset_weight += item.weight
        if subset_weight <= W and subset_value > best_value:
            best_value = subset_value
            knapsack = subset
    return knapsack


# uses a built-in sorting function comparing the weight to value
# ratio primarily and the value as a secondary comparison in the
# event of a tie
def get_sorted_ratios(input_list):
    return sorted(input_list, key=lambda x: (x.ratio, x.value))

# heuristic approach from pseudocode
def heuristic(W, n, items):
    knapsack = []
    currentW = W

    items_list = get_sorted_ratios(items)

    for item in items_list:
        if item.weight <= currentW:
            knapsack.append(item)
            currentW -= item.weight

    return knapsack
\end{verbatim}

\end{document} 
